\section{Related Work}
In 2010, \cite{hiemstra2010mapreduce} proposed the MapReduce framework for information retrieval. The solution proposed in this paper uses a method a lot like information retrieval, also based on the MapReduce framework: it generates a big ranked list of strings, which are extracted from a large dataset. However, this solution does not create a fulltext index, and it ranks by count rather than relevance to the search query.
For extraction of the JavaScript library locations, we use an approach similar as the one described in \cite{muhleisen2012web}. We use the path within the 'src' attribute of a script tag to extract the location and name of a JavaScript library.
Similar research is conducted in 2012 by \cite{nikiforakis2012libraries}. They partly address the same problem as discussed in this paper: the security issues that come with trusting a CDN. \cite{nikiforakis2012libraries} uses the top 10,000 sites from Alexa, whereas this paper uses the CommonCrawl dataset. The paper \cite{bojkovic2011cdn} explains how a content delivery network works, and why these are important.
